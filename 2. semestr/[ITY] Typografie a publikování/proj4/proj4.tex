\documentclass[a4paper,11pt]{article}
\usepackage[left=2cm,text={17cm, 24cm},top=3cm]{geometry}
\usepackage[utf8]{inputenc}
\usepackage[czech]{babel}
\usepackage[numbers]{natbib}
\usepackage{xurl}
\usepackage[hidelinks]{hyperref}
\DeclareUrlCommand\url{\def\UrlLeft{<}\def\UrlRight{>} \urlstyle{tt}}

\setcitestyle{number}

\begin{document}

\begin{titlepage}
\begin{center}
    
\textsc{\Huge Vysoké učení technické v Brně\\[0.4em]
\huge Fakulta informačních technologií} \\\vspace{\stretch{0.382}}
{\LARGE{Typografie a publikování -- 4. projekt}\\[0.3em] \Huge{Bibliografické údaje}}\\ \vspace{\stretch{0.618}}
\end{center}
{\Large \today \hfill Martin Chládek}

\end{titlepage}

\newpage

\section{Typografie}

Typografie je obor zabývající se uspořádáním prvků a textu v tisku. Disponuje především funkcí estetickou a komunikační, ale také funkcí uměleckou. Typografie, jako taková, vznikla současně se~vznikem knihtisku a zárověň s ním se i vyvíjela. K odlišení těchto pojmů došlo až s technickým rozvojem v~19.~století. Během tohoto milníku byla typografie modernizována a jejím primárním úkolem se stala úprava knih vyráběných ve velkých nákladech \cite{Marketa_Hoskova:Vojtech_Pressing_a_typografie_jeho_doby}.

Rozlišujeme dva druhy písma. Prvním typem je písmo neproporcionální, kde všechny znaky jsou shodně široké, s takovým písmem pracuje např. psací stroj. Na druhé straně stojí písmo proporcionální, kde každý znak má vlastní přiměřenou šířku. S takovým typem písma pracují především současné počítačové editory \cite{Lenka_Cerna:Typografie}. 

Dalším důležitým mezníkem při práci s písmem je vzdálenost mezi slovy. DTP\footnote{Desktop publishing -- tvorba tištěného dokumentu za použití počítače.} programy implicitně využívají základní mezislovní mezeru, která je třetinou stupně písma. Nastavením příslušné délky mezer lze docílit jiného celkového vzhledu tištěného dokumentu, ale taktéž může dojít k narušení jeho čitelnosti \cite{Vladimir_Beran:Typograficky_manual}.  

\section{Virtuální realita}

\emph{\uv{Zařízení virtuální reality dokáží vytvořit 3D scénu, ve které se člověk může pohybovat a pomocí zařízení pro interakci může tuto scénu měnit. Vše potom přispívá k umocnění zážitku dané situace.}\\(Bukvald,~2017, s.~15)} \cite{Josef_Bukvald:Studie_o_soucasnem_stavu_virtualni_reality}

Počátek stereoskopického vidění se datuje do 19. století, konkrétně do roku 1832, kdy Sir Charles Wheatstone vynalezl stereoskop. Tato technologie spočívala v zobrazování rozdílného obrazu do pravého a~levého oka. Tahle myšlenka položila základ moderním 3D televizím a také VR \cite{Jason_Jerald:The_VR_Book}.

Virtuální realita se ve 21. století dočkala velké pozornosti. Příčinou je téměř dvacet let výzkumu a~intenzivního vývoje, díky tomu se VR dostala do~stádia určité zralosti \cite{Ronald_Blach:Virtual_Reality_Technology}. V~současnosti se do~vývoje investují miliardy dolarů k jejímu zdokonalení \cite{TIME:Why_Virtual_Reality_Is_About_to_Change_the_World}. Nicméně technologie není zatím natolik vyvinutá, aby simulovala realitu zcela, a tak je kvůli latenci VR headsetů zaznamenávána spousta případů, kdy~uživatelům způsobily nevolnost \cite{BBC:Virtual_Reality_The_Complete_Guy}.

Velká část laického obyvatelstva se domnívá, že brýle pro virtuální realitu mají jediné využití ve~videohrách. Potenciál VR je ale podstatně větší. Využití nalézá například ve zdravotnictví, kde~chirurgové ve virtuální realitě mohou provádět operace užitím robotů, takže nemusí být na stejném místě jako pacient. Dále se dá využít ve vzdělávacích institucích nebo vojenských složkách \cite{Josef_Bukvald:Studie_o_soucasnem_stavu_virtualni_reality}. Hojně také pracují s VR piloti, kteří díky ní simulují nouzová přístání \cite{Michal_Chnoupek:Pocitace_IT}.

Mezi hlavní hráče na poli výrobců příslušenství pro VR patří např. společnost HTC s jejich headsetem \uv{HTC Vive Pro} a společnost Oculus pro jejich zařízení \uv{Oculus Rift} \cite{PC_Magazine:The_many_virtues_of_virtual_reality}.


\newpage
\bibliographystyle{czechiso}
\def\refname{Použitá literatura}
\bibliography{proj4}

\end{document}
